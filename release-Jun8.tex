\documentclass[11pt]{article}
\usepackage{color,xtab,suffix}
\usepackage{tikz}
%
% In these definitions, * means no space afterwards
\newcommand\Tag[1]{\Tag*{#1\space}}
% Change Tag* to alter the format of Tags
\WithSuffix\newcommand\Tag*[1]{{\small\sf #1}}
\WithSuffix\newcommand\Tagf*[1]{{\footnotesize\sf #1}}
% Major.Minor.#1
\newcommand\MMP[1]{\MMP*{#1\space}}
\WithSuffix\newcommand\MMP*[1]{\Tag*{Major.Minor.#1}}
% Major.Minor.Bugfix
\newcommand\MMBf{\MMBf*\space}
\WithSuffix\newcommand\MMBf*{\MMP*{Bugfix}}
% Major.Minor
\newcommand\MM{\MM*\space}
\WithSuffix\newcommand\MM*{\Tag*{Major.Minor}}
% Bugfix
\newcommand\Bf{\Bf*\space}
\WithSuffix\newcommand\Bf*{\Tag*{Bugfix}}
%
\title{Labelling Unicon Binary Releases}
\author{Don Ward \and Jafar Al Gharaibeh}
%\date{June 1{\small $^{\textstyle st}$} 2020}% ----- First public version
\date{June 8{\small $^{\textstyle th}$} 2020}% ----- Second public version
%\date{\today}
\begin{document}
\maketitle

Different versions of software must be labelled so that they may be readily
distinguished.  The purpose of any version labelling scheme is to bring some
order to a possible tangle of different revisions. Ideally, the scheme
should allow the user to determine quickly whether one release is earlier%
\footnote{
  We are not using ``earlier'' in a strict chronological sense here: one
  release might be earlier than another in developmental terms, even though
  it was released afterwards.
} than another.

\section*{A Scheme to Label Unicon Releases}
We propose using git tags with a structured name to identify and describe
Unicon releases. The tag name will have three parts described by three
numbers -- \MMBf*. \MM is the version of Unicon that is stored in
\texttt{src/h/version.h}; \Bf starts at \Tag*{0} and will increment by one on
each binary release of Unicon that has \MM as the first part: it will reset
to \Tag*{0} when \MM changes. \MMP{0} releases will normally be labelled as
\MM without the trailing \Tag*{.0}.

Although ``\Bf*'' implies that the reason for a subsequent release based on
\MM is merely to correct errors, it could also contain new features that are
not considered important enough to bump \MM but are desirable to release in
advance of the next \MM binary. There is some further discussion later on
about the implications of this.

As an aside, we considered using the form \MMP*{Patch} governed by the
popular Semantic Versioning specification [1] but decided against it because
the Unicon project is unlikely to follow the strict rules about when the
component numbers must alter. Our proposal is similar in spirit to [1] but
not as prescriptive.

The svn commit history was linear: after the transition from svn to git,
some effort is being made currently to ensure that commits made into the git
master branch continue to form a linear sequence by rebasing new pull
requests onto its tip. Rebasing involves some extra (manual, and possibly
error prone) steps during the commit process and, as we become more familar
with git, we may prefer to use merge commits instead -- which will mean a
departure from a strictly linear commit history -- although, if we zoom out,
the commit history will still appear linear in the large.  In any case, the
release numbering scheme and the release process itself must be able to deal
with a general graph (of commits) as well as a tree, or a simple sequence.

In the absence of maintenance releases, and even though the commit graph
itself may not be linear, we think that a zoomed out view that only has
releases in it should be linear, which leads to some rules about which
commits are eligible to build a release.  This is nothing more than
expecting the natural taxonomic ordering implied by \MMBf to be followed in
practice -- nobody would expect either \Tag{10.5.1} or \Tag{10.4.4} to be
earlier than \Tag*{10.4.3}. The rules are discussed in detail in an appendix.

We believe that ``Maintenance releases'', be they urgent bug fixes or
feature enhancements, will inevitably lead to the development of different
revisions of Unicon in parallel. The reason is that it is highly likely,
when we want to go back and fix something, that we will want to change just
that one thing and not bring along every change that has happened
subsequently -- i.e. we will not want to fix it by merging the necessary
modifications into the tip of the current master branch and releasing that.  Our
simple linear sequence of releases must become a tree. It {\em could} become
a general graph -- and git would happily cope with that -- but we are
proposing that it does not: maintenance releases should be on ``Long Term
Branches'' (LTBs) that we are {\bf never} going to merge back into the
master branch. In time, the release tree will thus come to resemble the
iconic Saguaro cactus. Note that the use of LTBs does not preclude
cherry-picking commits from the master branch into the LTB or vice-versa.

%% Perhaps a nice picture here, based on
\begin{verbatim}  
         (13.1.1)  (13.1.2)
             |         |                 (13.2.1)
          ---K---------R                     |
         /                               ----T
        /       --------F               /
       /       /         \             /
...   A---B---C---G---H---I---J---L---M---O---P---Q---S---U--- ...
      |                               |               |
   (13.1)                          (13.2)          (14.0)
\end{verbatim}
\begin{quote}
  Figure 1: An example release tree
\end{quote}


The \MMBf labelling scheme cannot handle two parallel developments with the
same \MM \underline{and} preserve the natural taxonomic ordering if both of
them result in a release. A binary release (of \MMP*{0}) thus defines the
{\em end} of the development of \MM on the master branch: the first act
after the release should be to increment \Tag{Minor} on the master
branch. So, in the figure above, B (and subsequent commits on the master
branch) would have a version of \Tag*{13.2}.

This brings up an interesting question: at what point between O and S may
the major version change to \Tag*{14}?  The answer is any of them. O (and its
descendants) would start off at \Tag{13.3} but they are all pre-release, so
the version may change at any point: i.e. a change in version number does
not trigger a release, whereas a release does trigger a change in version
number. Although the \Tag*{Major} version changed in the example given above,
there is no implication that a new release {\em must} alter the \Tag*{Major}
version number.

\section*{Displaying the Release Details}
The present system displays five separate pieces of information that can
help to classify and order different releases:
There is \MM and \texttt{Version Date}, all contained in the string returned
from \texttt{\&version}, and a ``repo-count'' combined with a git hash that
is returned as the \texttt{Revision} string from \texttt{\&features}. Only the
git hash uniquely identifies the revision, but it contains no clue as to
which revision is earlier when comparing two arbitrary builds.  When the
history is largely linear, as it is now, the ``repo-count'' is a good rough
and ready comparator but it might fail in the presence of a release tree (as
opposed to a release sequence).

We propose that the version string be derived automatically from the git
meta-data, based on the name of the branch -- ``\texttt{master}'' or
something else -- and whether or not the tip of the branch is tagged.

The version string will be derived using the following rules
\begin{enumerate}
\item The version date is the date of the HEAD commit.
\item If the HEAD commit has a tag name, that tag name is the version number.
\item If HEAD is untagged the version number depends on the branch name
\begin{enumerate}
\item The master branch will use a version number of
  \Tag{\$VERSION~(pre-release)}
\item Other branches will use the output of \texttt{git~describe},
  which will be something like TAG-n-hash (for example \Tag*{13.1-176-g943d2b72})
  
  \begin{tabular}{lp{9cm}}
    \Tag{13.1}      &   is the name of the most recent release tag
                           (starting from HEAD and going backwards
                           towards the root until a release tag is found).\\
     \Tag{176}     &   is the number of commits between HEAD and the tag.\\
     \Tag{943d2b7} &   is the short hash of the HEAD commit.
  \end{tabular}
\end{enumerate}
\end{enumerate}

In the table of example version strings below, the source code was exactly
the same: only the git meta-data changed for each build. Each version string
begins with ``\Tag{Unicon Version }'', which has been omitted from the
table.

\noindent
\begin{tabular}{ll|p{7cm}}
  Version ``number'' & Version Date & comments\\
  \hline
  \Tag{13.2 (pre-release)} &	  \Tag{07-Jun-2020} &
  Built on the master branch with HEAD untagged.\\
  \Tag{13.1-176-g943d2b72} &  \Tag{07-Jun-2020} &
  Built with the master branch renamed to something else
  (simulating a non-release build on a LTB).\\
  \Tag{13.1.1}	&  \Tag{07-Jun-2020} &
  Built with a tag of \Tag{13.1.1} pointing at HEAD
  (simulating a maintenance release build on a LTB).\\
  \Tag{13.1.1} & \Tag{07-Jun-2020} &
  Built on the master branch with a tag of \Tag{13.1.1} pointing at HEAD
  (demonstrates that a tag overrides the branch name).\\
\end{tabular}


Note that users who use binary releases will only ever see a version string
like \Tag*{Unicon Version 13.2 07-Jun-2020}. Users who build from the source
will usually see version strings like the first example (i.e. containing
``\Tag*{(pre-release)}'') The only time something like
\Tag{13.1-176-g943d2b72} will ever appear in the version string is if the
user has downloaded the latest source and has switched to another branch
before making a build. In such a case we assume they know enough to
unscramble the output of \texttt{git describe}.

When the compiler is in use, the string ``\Tag{(iconc)}'' is automatically
placed between the version number string and the date, as at present.

In future, should we decide to go down the release candidate route, we can
tag a release candidate with an appropriately named tag -- for example
\Tag{13.5-RC1} -- and the version string will automatically include the
information. Note that a release candidate would not be considered to be a
suitable starting point for an LTB%
\footnote{
  We have chosen to make all \Tagf*{Major.Minor} releases on the master
  branch but there is nothing fundamental about this. An alternative would
  be to start the LTB at \Tagf*{13.5-RC1} and then put \Tagf*{RC2} on that
  branch until, ultimately, \Tagf*{13.5} is released (on the LTB, rather
  than the master).  In such a scheme, the master branch would go to
  \Tagf*{13.6} immediately after the release of \Tagf*{13.5.0-RC1}.

  If a long time elapses between \Tagf*{13.5-RC1} and \Tagf*{13.5} this
  scheme might be preferable because it allows the development of
  \Tagf*{13.6} to proceed on the master branch without entangling it in the
  release of \Tagf*{13.5}. It's a decision we can defer until we decide to
  adopt the release candidate approach to releases.
  }%
; if there were a problem, we'd fix it by
by releasing \Tag{13.5-RC2} until, ultimately, \Tag{13.5} is released
and the master branch moves on to \Tag*{13.6}.

A non-release build will not have a release tag associated with it. The user
must make sense of the version strings (which include the git hash plus the
number of commits back to the last release) to decide which of two builds
is newer. But, to be in this situation in the the first place, the user must
have used git to retrieve the source to build it: we assume they are capable
of also interrogating git to establish the relative ordering of the two
builds -- if it isn't obvious -- should the need arise.

\subsection*{References:}

[1] Preston-Werner, Tom (2013). Semantic Versioning 2.0.0.\\
    \verb|http://semver.org/spec/v2.0.0.html|.

\newpage
\section*{Appendix: Choosing a release point}

If the commit graph is linear then it's simple: any commit will serve as a
release and the taxonomic ordering will automatically apply. If there are
branches in parallel with a potential release commit then some care must be
taken -- we should choose only one of the branches to make releases
from. Consider the graph below, which is taken from \verb|git log --help|.
We have drawn it twice to highlight the point that there is no ``main line''
between A and L, just descendants of A and parents of L.

\begin{verbatim}
                D---E-------F
               /     \       \
              B---C---G---H---I---J
             /                     \
            A-------K---------------L--M

              -----K---------------
             /                     \
            /   D---E-------F       \
           /   /     \       \       \
          A---B---C---G---H---I---J---L---M

 A < { K, B < { {D < E < F}, { C < G < H} } < I < J } < L < M
 E < G                     

 Consistent release combinations are a subset of
     {AKLM}            ( not {BCDRFGHIJ} )
 or
     {ABDEGHIJLM}      ( not {CFK} )
 or
     {ABDEFIJLM}       ( not {CGHK} )
 or
     {ABCGHIJLM}       ( not {DEFK} )
\end{verbatim}
\begin{quote}
  Figure 2: A commit graph
\end{quote}

To maintain the taxonomic ordering, we must pick release points from only
one of the alternatives listed. The easiest and least error-prone policy is
to make a release from a part of the commit graph that has nothing in
parallel with it and is thus, unambiguously, part of the main line. In the
diagram above, that would be one or more from \{ALM\}. The same rules apply
to a release from a LTB with the proviso that we only consider commits on
the LTB, rather than the master. If these rules are followed then the \MMBf
label describes the position of an individual release in the release tree
exactly.

\section*{Appendix: A Bestiary of Unicon Releases}
In this table, we identify the points at which the version details changed
and also (as far as is practicable) the points at which a binary release was
made. The table also contains suggestions to tag the historical releases.
Note that the historical ``\Bf*" releases before \Tag{13.1} do not
correspond with our naming proposals because they are in the main line
rather than on LTBs but, since there are no plans to make any maintenance
releases before \Tag{13.1.1} -- and, perhaps, not even then -- it doesn't
really matter.
%An alternative would be not to tag any release before \Tag{13.1.0}.

Windows Date is the Version Date reported by the Windows binary.

\vspace{0.5cm}
{\small
{\renewcommand{\arraystretch}{1.05} % Reduce line spacing to fit on two pages
%
% Define the table furniture.
%
\tablefirsthead{\hline%
  {\bf SVN} & {\bf git} & {\bf Version}%
  & {\bf Version} & {\bf Windows} & {\bf git}~~~~~~\\
  {\bf rev} & {\bf hash} &
  & {\bf Date} & {\bf Date} & {\bf tag}~~~~~~\\\hline}
\tablehead{\hline%
  {\bf SVN} & {\bf git} & {\bf Version}%
  & {\bf Version} & {\bf Windows} & {\bf git}~~~~~~\\
  {\bf rev} & {\bf hash} &
  & {\bf Date} & {\bf Date} & {\bf tag}~~~~~~\\\hline}
\tabletail{\multicolumn{5}{r}{\small continued \ldots}\\}
\tablelasttail{\hline}
% the default value (of 0.1) gives very short pages: this works better for this table.
\xentrystretch{-0.15}
%
\definecolor{lightgrey}{RGB}{180,180,180}
\newcommand{\gr}[1]{\color{lightgrey}#1}
\noindent
\begin{xtabular}{|cc@{\hspace{0.5cm}}lll@{\hspace{0.5cm}}l|}
%%
%% In reverse chronological order ----------------------------------------
%%
%%SVN& git hash & Version            & Version Date         & Win Date & git tag\\
     & a37d2288 & \gr{13.1}          & November 6, 2019     &&\\
     & f22d949a & \gr{13.1}          & August 19, 2019      &&\\
5965 & dce80670 & \gr{13.1}          & \gr{March 2, 2019}   & March 24, 2019 & \Tag*{13.1}\\
5920 & 283f5909 & \gr{13.1}          & March 2, 2019        &&\\
5888 & c27495fd & 13.1               & January 19, 2019     &&\\
5854 & 5eb096f4 & \gr{13.0}          & December 6, 2018     &&\\
5599 & a6a53187 & \gr{13.0}          & August 10, 2017      &&\\
5080 & 380c9ec5 & \gr{13.0}          & \gr{April 16, 2017}  & April 16, 2017 & \Tag*{13.0.1}\\
5076 & 2f471f57 & \gr{13.0}          & April 16, 2017       &&\\
5039 & 8ddc1399 & \gr{13.0}          & April 10, 2017       &&\\
4777 & 37b670df & \gr{13.0}          & Feb 1, 2017          & Feb 1, 2017 & \Tag*{13.0}\\
4768 & fad1e049 & \gr{13.0}          & Jan 30, 2017         &&\\
4500 & 3de78ac1 & 13.0               & Aug 30, 2016         &&\\
4237 & e9bb5587 & \gr{12.3}          & Feb 29, 2016         & Feb 29, 2016 & \Tag*{12.3}\\
4232 & 8cb66e3a & \gr{12.3}          & Feb 29, 2016         &&\\
4166 & e717f130 & 12.3               & Sep 12, 2015         && \\
3976 & 2edead79 & 12.2               & \gr{Dec 2, 2014}     & Dec 2, 2014 & \Tag*{12.2}\\
3975 & 2b1344d4 & 12.3               & Dec 2, 2014          &&\\
3883 & 250626b8 & \gr{12.1}          & July 17, 2014        &&\\
3420 & eee39677 & \gr{12.1}          & April 13, 2013       & April 13, 2013 & \Tag*{12.1.2}\\
3312 & a89fd52c & \gr{12.1}          & November 19, 2012    &&\\
3231 & 4ae6595f &\gr{12.1}           &\gr{August 06, 2012}  & August 12, 2012 & \Tag*{12.1.1}\\
3211 & 7f7f54e6 & \gr{12.1}          & August 06, 2012      & August 6, 2012 & \Tag*{12.1}\\
3101 & 3dc88528 & \gr{12.1}          & May 31, 2012         &&\\
3090 & 509e849d & 12.1               & May 21, 2012         &&\\
2983 & 9db42528 & \gr{12.0 or 11.9}  &\gr{November 2, 2011} & November 2, 2011 & \Tag*{12.0.3}\\
2857 & 6c12273b & \gr{12.0 or 11.9}  & November 2, 2011     & November 2, 2011 & \Tag*{12.0.2}\\
2788 & bdffe5e7 & 12.0 or 11.9       & July 15, 2011        & July 15, 2011 & \Tag*{11.9.1}\\
2786 & 39258419 & \gr{11.7}          &\gr{January 14, 2011} & July 13, 2011 & \Tag*{12.0.1}\\
2708 & 547d0670 & \gr{11.7}          &\gr{January 14, 2011} & March 10, 2011 & \Tag*{12.0} \Tag*{11.9}\\
2675 & 11466f2e & \gr{11.7}          &\gr{January 14, 2011} & February 22, 2011 & \Tag*{11.8}\\
2623 & 39a3e916 & \gr{11.7}          & January 14, 2011     &&\\
2535 & 268a2a6a & \gr{11.7}          &\gr{January 22, 2010} & January 22, 2010 & \Tag*{11.7.1}\\
2237 & 7465a6d4 & 11.7               & January 22, 2010     & January 22, 2010 & \Tag*{11.7}\\
2094 & 14bf54ab & 11.6               & October 12, 2009     &&\\
1932 & 5ee2eb2f & \gr{11.5}          & March 22, 2009       &&\\
1636 & a683b0ad & 11.5               & April 27, 2008       &&\\
1362 & 5b279165 & 11.4               & January 10, 2007     &&\\
1273 & 018bbb2a & 11.3               & March 20, 2006       &&\\
1213 & 5c306bdc & \gr{11.3 (beta)}   & November 30, 2005    &&\\
1117 & 85d2d3f2 & \gr{11.3 (beta)}   & June 23, 2005        &&\\
1078 & 95d17114 & 11.3 (beta)        & April 28, 2005       &&\\
1051 & 1eb51620 & \gr{11.2 (beta)}   & February 18, 2005    &&\\
1018 & 4e83fa58 & 11.2 (beta)        & January 4, 2005      &&\\
922  & c73adaed & \gr{11.1 (beta)}   & October 18, 2004     &&\\
882  & 0df3e550 & 11.1 (beta)        & July 6, 2004         &&\\
810  & 794f8018 & \gr{11.0 (beta)}   & May 15, 2004         &&\\
672  & 520e9aea & \gr{11.0 (beta)}   & February 1, 2004     &&\\
548  & 63b5dcc1 & \gr{11.0 (beta)}   & July 8, 2003         &&\\
520  & 175cc419 & \gr{11.0 (beta)}   & June 2, 2003         &&\\
492  & d7388853 & \gr{11.0 (beta)}   & May 24, 2003         &&\\
484  & 11eea964 & 11.0 (beta)        & May 23, 2003         &&\\
411  & bf6174b2 & \gr{10.0 (beta-3)} & November 25, 2002    &&\\
395  & f7804ff1 & \gr{10.0 (beta-3)} & October 25, 2002     &&\\
369  & a5cadd88 & \gr{10.0 (beta-3)} & August 24, 2002      &&\\
274  & 4ea85e6f & \gr{10.0 (beta-3)} & April 27, 2002       &&\\
244  & 441aad90 & 10.0 (beta-3)      & February 14, 2002    &&\\
58   & be738a6a & 10.0 (beta-2)      & August 1, 2001       &&\\
3    & 984fa078 & 10.0 beta          & January 18, 2001     &&\\
\end{xtabular}
} % \arraystretch is back to its previous value
%
% Remove the table furniture.
% (without this, the next table is typeset with these definitions!)
%
\tablefirsthead{}
\tablehead{}
\tabletail{}
\tablelasttail{}
\xentrystretch{0.1}
%
}% end /small


\end{document}
